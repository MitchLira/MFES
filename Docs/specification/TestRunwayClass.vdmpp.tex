\begin{vdmpp}[breaklines=true]
class TestRunwayClass is subclass of MyTestCase

instance variables
-- TODO Define instance variables here
 --e1: Runway := new Event("nome",mk_Date(year, month, day),"place","theme",price, MaxSpectators);
 -- fashion shows
 f1: Runway := new Runway("Wonderland", mk_Platform`Date(2018, 9, 20),"London","Fantasy",75,100);
 f2: Runway := new Runway("New World", mk_Platform`Date(2019, 11, 10),"U.S.A", "Rock",100,60);
 f3: Runway := new Runway("Pop Culture", mk_Platform`Date(2018, 8, 2),"Paris", "Pop",20,90);
 
 --designers
 d1: Designer := new Designer("Miguel Lira");
 d2: Designer := new Designer("Miriam Gonçalves");
 d3: Designer := new Designer("Paulo Sérgio");
 d4: Designer := new Designer("Coco Chanel");
 d5: Designer := new Designer("Ralph Lauren");
 
 --models
 m1: Model:= new Model("Adriana Lima", 36, 1.78, "Brasil", <Female>);
 m2: Model:= new Model("Sara Sampaio", 26, 1.72, "Portugal", <Female>);
 m3: Model:= new Model("Karlie Kloss", 25, 1.88, "U.S.A", <Female>);
 m4: Model:= new Model("Gigi Hadid", 22, 1.79, "U.S.A", <Female>);
 m5: Model:= new Model("Candice Swanepoel", 29, 1.77, "Africa do Sul", <Female>);
 m6: Model:= new Model("Lily Aldridge", 32, 1.75, "U.S.A", <Female>);
 m7: Model:= new Model("Ashley Graham", 30, 1.75, "U.S.A", <Female>);
 m8: Model:= new Model("Miles McMillan", 28, 1.88, "U.S.A", <Male>);
 
 -- items
 it1: Item := new Item("Camisolinha de la","1c34ff445",220.50,<XL>);
 it2: Item := new Item("Oculos de Sol Gucci","123ggg4hk",220.50,<S>);
 it3: Item := new Item("Calcinha Branca","1c34ff445",220.50,<M>);
 it4: Item := new Item("Camisola Sarja Preta Versace","3213fff23",220.50,<L>);
 it5: Item := new Item("Camisolinha de la","1c34ff445",220.50,<S>);
 it6: Item := new Item("Blusa axadrezada","1c34ff345",203,<XS>);
 it7: Item := new Item("Calcas rasgadas","2c34ff445",220,<S>);
 it8: Item := new Item("Camisa Rosa","1c32ff445",120,<M>);
operations
(*@
\label{testRunwayAttributes:38}
@*)
 public testRunwayAttributes: () ==> ()
 testRunwayAttributes() == (
  IO`println("\t (1) Construtor de um Runway");
  assertEqual(f1.name, "Wonderland");
  assertEqual(f1.date, mk_Platform`Date(2018, 9, 20));
  assertEqual(f1.place, "London");
  assertEqual(f1.theme, "Fantasy");
  assertEqual(f1.price, 75);
  assertEqual(f1.maxSpectators, 100);
 );
 
(*@
\label{testAddModel:49}
@*)
 public testAddModel: () ==> ()
 testAddModel() == (
  IO`println("\t (2) Adicao de uma modelo a um desfile");
    f1.setModels({m1,m2,m3});
    assertEqual(f1.models, {m1,m2,m3});
    f1.addModel(m4);
    assertEqual(f1.models,{m1,m2,m3,m4});
 
 );
 
(*@
\label{testAddModels:59}
@*)
 public testAddModels: () ==> ()
 testAddModels() == (
  IO`println("\t (3) Adicao de um conjunto de modelos a um desfile");
   let d1 = new Runway("Wonderland", mk_Platform`Date(2018, 9, 20),"London","Fantasy",75,100) in (
    d1.setModels({m1});
    assertEqual(d1.models, {m1});
    d1.addModels({m4, m2, m3, m5});
    assertEqual(d1.models,{m1,m4,m2,m3,m5});
    d1.addModels({m2,m3,m6});
    assertEqual(d1.models,{m1,m4,m2,m3,m5,m6});
  );
 );
 
(*@
\label{testRemModel:72}
@*)
 public testRemModel: () ==> ()
 testRemModel() == (
  IO`println("\t (4) Remocao de uma modelo de um desfile");
   let d1 = new Runway("Wonderland", mk_Platform`Date(2018, 9, 20),"London","Fantasy",75,100) in (
    d1.setModels({m1,m2,m3});
    assertEqual(d1.models, {m1,m2,m3});
    d1.remModel(m3);
    assertEqual(d1.models,{m1,m2});
  );
 );
 
(*@
\label{testRemModels:83}
@*)
  public testRemModels: () ==> ()
  testRemModels() == (
  IO`println("\t (5) Remocao de um conjunto de modelos de um desfile");
   let d1 = new Runway("Wonderland", mk_Platform`Date(2018, 9, 20),"London","Fantasy",75,100) in (
    d1.setModels({m1, m2 , m3});
    assertEqual(d1.models, {m1, m2, m3});
    d1.remModels({m2,m3});
    assertEqual(d1.models,{m1});
  );
 );
 
(*@
\label{testAddDesigner:94}
@*)
 public testAddDesigner: () ==> ()
 testAddDesigner() == (
  IO`println("\t (6) Adicao de um designer e os seus items a um desfile");
  let show1 = new Runway("Wonderland", mk_Platform`Date(2018, 9, 20),"London","Fantasy",75,100) in (
    d1.addItems({it1, it2});
    show1.addDesigner(d1);
    assertEqual(show1.designers, {d1});
    assertEqual(show1.expositionItems, {d1|->{it1,it2}});
  );
 );
 
(*@
\label{testRemDesigner:105}
@*)
 public testRemDesigner: () ==> ()
 testRemDesigner() == (
  IO`println("\t (7) Remocao de um designer e dos seus items de um desfile");
  let d6: Designer = new Designer("Karl Lagerfeld"),
    d7: Designer = new Designer("Donatella Versace"),
  show1 = new Runway("Wonderland", mk_Platform`Date(2018, 9, 20),"London","Fantasy",75,100) in (
    d6.addItems({it1, it2});
    d7.addItems({it4,it5});
    show1.addDesigner(d6);
    assertEqual(show1.designers, {d6});
    assertEqual(show1.expositionItems, {d6|->{it1,it2}});
    show1.addDesigner(d7);
    assertEqual(show1.designers, {d6,d7});
    assertEqual(show1.expositionItems, {d6|->{it1,it2},d7|->{it4,it5}});
    show1.removeDesigner(d6);
    assertEqual(show1.designers,{d7});
    assertEqual(show1.expositionItems, {d7|->{it4,it5}})
  );
 );
 
(*@
\label{testItemsOfDesigner:125}
@*)
 public testItemsOfDesigner: () ==> ()
 testItemsOfDesigner() == (
  IO`println("\t (8) Selecao de items de um designer espec�fico de um desfile");
  let d6: Designer = new Designer("Karl Lagerfeld"),
    d7: Designer = new Designer("Donatella Versace"),
  show1 = new Runway("Wonderland", mk_Platform`Date(2018, 9, 20),"London","Fantasy",75,100) in (
    d6.addItems({it1, it2});
    d7.addItems({it4,it5});
    show1.addDesigner(d6);
    assertEqual(show1.designers, {d6});
    assertEqual(show1.expositionItems, {d6|->{it1,it2}});
    show1.addDesigner(d7);
    assertEqual(show1.designers, {d6,d7});
    assertEqual(show1.expositionItems, {d6|->{it1,it2},d7|->{it4,it5}});
    assertEqual(show1.getItemsOfDesignerInShow(d6), {it1,it2});
  );
 );
 
(*@
\label{testAddDesignerItem:143}
@*)
 public testAddDesignerItem: () ==> ()
 testAddDesignerItem() == (
  IO`println("\t (9) Adicao de um item a um designer de um desfile");
  let d6: Designer = new Designer("Karl Lagerfeld"),
    d7: Designer = new Designer("Donatella Versace"),
  show1 = new Runway("Wonderland", mk_Platform`Date(2018, 9, 20),"London","Fantasy",75,100) in (
    d6.addItems({it1, it2});
    d7.addItems({it4});
    show1.addDesigner(d6);
    assertEqual(show1.designers, {d6});
    assertEqual(show1.expositionItems, {d6|->{it1,it2}});
    show1.addDesigner(d7);
    assertEqual(show1.designers, {d6,d7});
    assertEqual(show1.expositionItems, {d6|->{it1,it2},d7|->{it4}});
    assertEqual(show1.getItemsOfDesignerInShow(d6), {it1,it2});
    show1.addDesignerItem(d6, it5);
    assertEqual(show1.expositionItems, {d6|->{it1,it2,it5},d7|->{it4}});
    assertEqual(show1.getItemsOfDesignerInShow(d6), {it1,it2,it5});
  );
 );
 
(*@
\label{testRemDesignerItem:164}
@*)
 public testRemDesignerItem: () ==> ()
 testRemDesignerItem() == (
  IO`println("\t (10) Remocao de um item de um designer de um desfile");
  let d6: Designer = new Designer("Karl Lagerfeld"),
    d7: Designer = new Designer("Donatella Versace"),
  show1 = new Runway("Wonderland", mk_Platform`Date(2018, 9, 20),"London","Fantasy",75,100) in (
    d6.addItems({it1, it2});
    d7.addItems({it4,it5,it6});
    show1.addDesigner(d6);
    assertEqual(show1.designers, {d6});
    assertEqual(show1.expositionItems, {d6|->{it1,it2}});
    show1.addDesigner(d7);
    assertEqual(show1.designers, {d6,d7});
    assertEqual(show1.expositionItems, {d6|->{it1,it2},d7|->{it4,it5,it6}});
    assertEqual(show1.getItemsOfDesignerInShow(d6), {it1,it2});
    show1.removeDesignerItem(d7, it5);
    assertEqual(show1.expositionItems, {d6|->{it1,it2},d7|->{it4,it6}});
    assertEqual(show1.getItemsOfDesignerInShow(d7), {it4,it6});
  );
 );
 
(*@
\label{testItemsInShow:185}
@*)
 public testItemsInShow: () ==> ()
 testItemsInShow() == (
  IO`println("\t (11) Selecao de items de um desfile");
  let d6: Designer = new Designer("Karl Lagerfeld"),
    d7: Designer = new Designer("Donatella Versace"),
  show1 = new Runway("Wonderland", mk_Platform`Date(2018, 9, 20),"London","Fantasy",75,100) in (
    d6.addItems({it1, it2});
    d7.addItems({it4,it5,it6});
    show1.addDesigner(d6);
    assertEqual(show1.designers, {d6});
    assertEqual(show1.expositionItems, {d6|->{it1,it2}});
    show1.addDesigner(d7);
    assertEqual(show1.designers, {d6,d7});
    assertEqual(show1.expositionItems, {d6|->{it1,it2},d7|->{it4,it5,it6}});
    assertEqual(show1.getItemsOfDesignerInShow(d6), {it1,it2});
    assertEqual(show1.getItemsInShow(), {it1,it2,it4,it5,it6});
  );
 );
 
(*@
\label{testSetModelItem:204}
@*)
 public testSetModelItem: () ==> ()
 testSetModelItem() == (
  IO`println("\t (12) Adicionar um item a uma modelo num desfile");
  let d6: Designer = new Designer("Karl Lagerfeld"),
    d7: Designer = new Designer("Donatella Versace"),
  show1 = new Runway("Wonderland", mk_Platform`Date(2018, 9, 20),"London","Fantasy",75,100) in (
    d6.addItems({it1, it2});
    d7.addItems({it4,it5,it6});
    show1.addDesigner(d6);
    show1.addModels({m1, m2});
    show1.addDesigner(d7);
    assertEqual(show1.modelsItems,{m1|->{},m2|->{}});
    show1.setModelItem(m1, it1);
    assertEqual(show1.modelsItems,{m1|->{it1},m2|->{}});
    show1.setModelItem(m2, it6);
    assertEqual(show1.modelsItems,{m1|->{it1},m2|->{it6}});
  );
 );
 
(*@
\label{testSetModelOutfit:223}
@*)
 public testSetModelOutfit: () ==> ()
 testSetModelOutfit() == (
  IO`println("\t (13) Adicao de um conjunto de items a uma modelo de um desfile");
  let d6: Designer = new Designer("Karl Lagerfeld"),
    d7: Designer = new Designer("Donatella Versace"),
  show1 = new Runway("Wonderland", mk_Platform`Date(2018, 9, 20),"London","Fantasy",75,100) in (
    d6.addItems({it1, it2});
    d7.addItems({it4,it5,it6});
    show1.addDesigner(d6);
    show1.addModels({m1, m2});
    show1.addDesigner(d7);
    assertEqual(show1.modelsItems,{m1|->{},m2|->{}});
    show1.setModelOutfit(m1, {it1,it5});
    assertEqual(show1.modelsItems,{m1|->{it1,it5},m2|->{}});
    show1.setModelOutfit(m2, {it6,it2,it4});
    assertEqual(show1.modelsItems,{m1|->{it1,it5},m2|->{it6,it2,it4}});
  );
 );
 
 
(*@
\label{testRemModelOutfit:243}
@*)
 public testRemModelOutfit: () ==> ()
 testRemModelOutfit() == (
  IO`println("\t (14) Remocao de um conjunto de items de uma modelo num desfile");
  let d6: Designer = new Designer("Karl Lagerfeld"),
    d7: Designer = new Designer("Donatella Versace"),
  show1 = new Runway("Wonderland", mk_Platform`Date(2018, 9, 20),"London","Fantasy",75,100) in (
    d6.addItems({it1, it2});
    d7.addItems({it4,it5,it6});
    show1.addDesigner(d6);
    show1.addModels({m1, m2});
    show1.addDesigner(d7);
    assertEqual(show1.modelsItems,{m1|->{},m2|->{}});
    show1.setModelOutfit(m1, {it1,it5});
    show1.setModelOutfit(m2, {it6,it2,it4});
    assertEqual(show1.modelsItems,{m1|->{it1,it5},m2|->{it6,it2,it4}});
    show1.removeModelOutfit(m1);
  assertEqual(show1.modelsItems,{m1|->{},m2|->{it6,it2,it4}});
  show1.removeModelOutfit(m2);
  assertEqual(show1.modelsItems,{m1|->{},m2|->{}});    
  );
 );
 
(*@
\label{testRemModelItem:265}
@*)
 public testRemModelItem: () ==> ()
 testRemModelItem() == (
  IO`println("\t (15) Remocao de um item de uma modelo num desfile");
  let d6: Designer = new Designer("Karl Lagerfeld"),
    d7: Designer = new Designer("Donatella Versace"),
  show1 = new Runway("Wonderland", mk_Platform`Date(2018, 9, 20),"London","Fantasy",75,100) in (
    show1.addDesigner(d6);
    show1.addModels({m1, m2});
    show1.addDesigner(d7);
    assertEqual(show1.modelsItems,{m1|->{},m2|->{}});
    show1.setModelOutfit(m1, {it1,it5});
    show1.setModelOutfit(m2, {it6,it2,it4});
    assertEqual(show1.modelsItems,{m1|->{it1,it5},m2|->{it6,it2,it4}});
  show1.removeModelItem(m1, it1);
  assertEqual(show1.modelsItems,{m1|->{it5},m2|->{it6,it2,it4}});
  );
 );
 
(*@
\label{testAll:283}
@*)
 public testAll: () ==> ()
 testAll() == (
 IO`println("Testes da classe Runway:");
  testRunwayAttributes();
  testAddModel();
  testAddModels();
  testRemModel();
  testRemModels();
  testAddDesigner();
  testRemDesigner();
  testItemsOfDesigner();
  testAddDesignerItem();
  testRemDesignerItem();
  testItemsInShow();
  testSetModelItem();
  testSetModelOutfit();
  testRemModelOutfit();
  testRemModelItem();
 );

end TestRunwayClass
\end{vdmpp}
\bigskip
\begin{longtable}{|l|r|r|r|}
\hline
Function or operation & Line & Coverage & Calls \\
\hline
\hline
\hyperref[testAddDesigner:94]{testAddDesigner} & 94&100.0\% & 1 \\
\hline
\hyperref[testAddDesignerItem:143]{testAddDesignerItem} & 143&100.0\% & 1 \\
\hline
\hyperref[testAddModel:49]{testAddModel} & 49&100.0\% & 1 \\
\hline
\hyperref[testAddModels:59]{testAddModels} & 59&100.0\% & 1 \\
\hline
\hyperref[testAll:283]{testAll} & 283&100.0\% & 1 \\
\hline
\hyperref[testItemsInShow:185]{testItemsInShow} & 185&100.0\% & 1 \\
\hline
\hyperref[testItemsOfDesigner:125]{testItemsOfDesigner} & 125&100.0\% & 1 \\
\hline
\hyperref[testRemDesigner:105]{testRemDesigner} & 105&100.0\% & 2 \\
\hline
\hyperref[testRemDesignerItem:164]{testRemDesignerItem} & 164&100.0\% & 1 \\
\hline
\hyperref[testRemModel:72]{testRemModel} & 72&100.0\% & 2 \\
\hline
\hyperref[testRemModelItem:265]{testRemModelItem} & 265&100.0\% & 1 \\
\hline
\hyperref[testRemModelOutfit:243]{testRemModelOutfit} & 243&100.0\% & 1 \\
\hline
\hyperref[testRemModels:83]{testRemModels} & 83&100.0\% & 1 \\
\hline
\hyperref[testRunwayAttributes:38]{testRunwayAttributes} & 38&100.0\% & 1 \\
\hline
\hyperref[testSetModelItem:204]{testSetModelItem} & 204&100.0\% & 1 \\
\hline
\hyperref[testSetModelOutfit:223]{testSetModelOutfit} & 223&100.0\% & 1 \\
\hline
\hline
TestRunwayClass.vdmpp & & 100.0\% & 18 \\
\hline
\end{longtable}

