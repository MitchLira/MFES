\begin{vdmpp}[breaklines=true]
class TestModelClass is subclass of MyTestCase
instance variables
 --designers
 d1: Designer := new Designer("Oscar de La Renta");
 d2: Designer := new Designer("Donna Karen");
 d3: Designer := new Designer("Alexander McQueen");
 d4: Designer := new Designer("Coco Chanel");
 d5: Designer := new Designer("Ralph Lauren");
 d6: Designer := new Designer("Karl Lagerfeld");
 d7: Designer := new Designer("Donatella Versace");
 
 --e1: Runway := new Event("nome",mk_Date(year, month, day),"place","theme",price, MaxSpectators);
 -- fashion shows
 f1: Runway := new Runway("Wonderland", mk_Platform`Date(2018, 9, 20),"London","Fantasy",75,100);
 f2: Runway := new Runway("New World", mk_Platform`Date(2019, 11, 10),"U.S.A", "Rock",100,60);
 f3: Runway := new Runway("Pop Culture", mk_Platform`Date(2018, 8, 2),"Paris", "Pop",20,90);
 f4: Runway := new Runway("Angels", mk_Platform`Date(2018,3, 1),"Paris", "Fantasy",200,50);
 f5: Runway := new Runway("Wonderland", mk_Platform`Date(2018, 9,21 ),"London", "Fantasy",120,40);
 f6: Runway := new Runway("Wonderland", mk_Platform`Date(2019, 12, 17),"London", "Fantasy",30,100);
 f7: Runway := new Runway("Wonderland", mk_Platform`Date(2020, 12, 17),"London", "Fantasy",40,120);
 
operations
(*@
\label{testGetModelsAttributes:23}
@*)
 public testGetModelsAttributes: () ==> ()
 testGetModelsAttributes() == (
    IO`println("\t (1) Constru��o de um Model");
   
    let m1 = new Model("Adriana Lima", 36, 1.78, "Brasilian", <Female>) in (
   assertEqual(m1.name,"Adriana Lima");
   assertEqual(m1.age, 36);
   assertEqual(m1.gender, <Female>);
   assertEqual(m1.height,1.78);
   assertEqual(m1.nationality, "Brasilian");
   );
 );
 
(*@
\label{testSetShowsModels:36}
@*)
 public testSetShowsModels: () ==> ()
 testSetShowsModels() == (
  IO`println("\t (2) Altera��o de um conjunto de shows de um Model");
    let m1 = new Model("Adriana Lima", 36, 1.78, "Brasilian", <Female>) in (
   m1.setShows({f1, f2, f3});
   assertEqual(m1.shows, {f1,f2,f3});
   );
 ); 
 
(*@
\label{testAddShowModels:45}
@*)
 public testAddShowModels: () ==> ()
 testAddShowModels() == (
    IO`println("\t (3) Adi��o de um show a um Model");
    let m1 = new Model("Adriana Lima", 36, 1.78, "Brasilian", <Female>) in (
   m1.setShows({f1, f2, f4});
   assertEqual(m1.shows, {f1,f2,f4});
   m1.addShow(f5);
   assertEqual(m1.shows,{f1,f2,f4,f5});
   m1.addShow(f6);
   assertEqual(m1.shows,{f1,f2,f4,f5,f6});
   m1.addShow(f7);
   assertEqual(m1.shows,{f1,f2,f4,f5,f6,f7});
   );
 ); 
 
(*@
\label{testRemShowModels:60}
@*)
 public testRemShowModels: () ==> ()
 testRemShowModels() == (
    IO`println("\t (4) Remo��o de um show de um Model");
    let m1 = new Model("Adriana Lima", 36, 1.78, "Brasilian", <Female>) in (
   m1.setShows({f1, f2, f4});
   assertEqual(m1.shows, {f1,f2,f4});
   m1.remShow(f2);
   assertEqual(m1.shows,{f1,f4});
   );
 );
 
 -- Entry point that runs all tests with valid inputs
(*@
\label{testAll:72}
@*)
  public testAll: () ==> ()
  testAll() == (
  IO`println("Testes da classe Model:");
   testGetModelsAttributes();
   testSetShowsModels();
   testAddShowModels();
   testRemShowModels();
  );
end TestModelClass
\end{vdmpp}
\bigskip
\begin{longtable}{|l|r|r|r|}
\hline
Function or operation & Line & Coverage & Calls \\
\hline
\hline
\hyperref[testAddShowModels:45]{testAddShowModels} & 45&100.0\% & 3 \\
\hline
\hyperref[testAll:72]{testAll} & 72&100.0\% & 3 \\
\hline
\hyperref[testGetModelsAttributes:23]{testGetModelsAttributes} & 23&100.0\% & 3 \\
\hline
\hyperref[testRemShowModels:60]{testRemShowModels} & 60&100.0\% & 3 \\
\hline
\hyperref[testSetShowsModels:36]{testSetShowsModels} & 36&100.0\% & 3 \\
\hline
\hline
TestModelClass.vdmpp & & 100.0\% & 15 \\
\hline
\end{longtable}

