\begin{vdmpp}[breaklines=true]
class TestDesignerClass is subclass of MyTestCase
instance variables
 -- items
 it1: Item := new Item("Camisolinha de la","1c34ff445",220.50,<XL>);
 it2: Item := new Item("Oculos de Sol Gucci","123ggg4hk",220.50,<S>);
 it3: Item := new Item("Calcinha Branca","1c34ff445",220.50,<M>);
 it4: Item := new Item("Camisola Sarja Preta Versace","3213fff23",220.50,<L>);
 it5: Item := new Item("Camisolinha de la","1c34ff445",220.50,<XS>);
 it6: Item := new Item("Camisolinha de la","1c34ff445",220.50,<XS>);
 it7: Item := new Item("Camisolinha de la","1c34ff445",220.50,<S>);
 it8: Item := new Item("Camisolinha de la","1c34ff445",220.50,<M>);
 
 -- models
 m1: Model:= new Model("Adriana Lima", 36, 1.78, "Brasilian", <Female>);
 m2: Model:= new Model("Sara Sampaio", 26, 1.72, "Portuguese", <Female>);
 m3: Model:= new Model("Karlie Kloss", 25, 1.88, "American", <Female>);
 m4: Model:= new Model("Gigi Hadid", 22, 1.79, "American", <Female>);
 m5: Model:= new Model("Candice Swanepoel", 29, 1.77, "African", <Female>);
 m6: Model:= new Model("Lily Aldridge", 32, 1.75, "American", <Female>);
 m7: Model:= new Model("Ashley Graham", 30, 1.75, "American", <Female>);
 m8: Model:= new Model("Miles McMillan", 28, 1.88, "American", <Male>);

operations 
(*@
\label{testAddItem:24}
@*)
 public testAddItem: () ==> ()
 testAddItem() == (
  IO`println("\t (1) Adicao de um item a um designer");
   let d1 = new Designer("Coco Chanel") in (
    d1.setItems({it1,it2,it3});
    assertEqual(d1.items, {it1,it2,it3});
    d1.addItem(it4);
    assertEqual(d1.items,{it1,it2,it3,it4});
  );
 );
 
(*@
\label{testAddItems:35}
@*)
 public testAddItems: () ==> ()
 testAddItems() == (
  IO`println("\t (2) Adicao de um conjunto de items a um designer");
   let d1 = new Designer("Coco Chanel") in (
    d1.setItems({it1});
    assertEqual(d1.items, {it1});
    d1.addItems({it4, it2, it3, it5});
    assertEqual(d1.items,{it1,it4,it2,it3,it5});
    d1.addItems({it2,it3,it6});
    assertEqual(d1.items,{it1,it4,it2,it3,it5,it6});
  );
 );
 
(*@
\label{testRemItem:48}
@*)
 public testRemItem: () ==> ()
 testRemItem() == (
  IO`println("\t (4) Remocao de um item de um designer");
   let d1 = new Designer("Coco Chanel") in (
    d1.setItems({it1,it2,it3});
    assertEqual(d1.items, {it1,it2,it3});
    d1.remItem(it3);
    assertEqual(d1.items,{it1,it2});    
  );
 );
 
(*@
\label{testRemItems:59}
@*)
 public testRemItems: () ==> ()
 testRemItems() == (
  IO`println("\t (3) Remocao de um conjunto de items de um designer");
   let d1 = new Designer("Coco Chanel") in (
    d1.setItems({it1, it2 , it3});
    assertEqual(d1.items, {it1, it2, it3});
    d1.remItems({it2,it3});
    assertEqual(d1.items,{it1});
  );
 );
 

 
 
 -- Entry point that runs all tests with valid inputs
(*@
\label{testAll:74}
@*)
  public testAll: () ==> ()
  testAll() == (
  IO`println("Testes da classe Designer:");
    testAddItem();
    testAddItems();
    testRemItems();
    testRemItem();
    
  );
 
end TestDesignerClass
\end{vdmpp}
\bigskip
\begin{longtable}{|l|r|r|r|}
\hline
Function or operation & Line & Coverage & Calls \\
\hline
\hline
\hyperref[testAddItem:24]{testAddItem} & 24&100.0\% & 1 \\
\hline
\hyperref[testAddItems:35]{testAddItems} & 35&100.0\% & 1 \\
\hline
\hyperref[testAll:74]{testAll} & 74&100.0\% & 1 \\
\hline
\hyperref[testRemItem:48]{testRemItem} & 48&100.0\% & 1 \\
\hline
\hyperref[testRemItems:59]{testRemItems} & 59&100.0\% & 1 \\
\hline
\hline
TestDesignerClass.vdmpp & & 100.0\% & 5 \\
\hline
\end{longtable}

